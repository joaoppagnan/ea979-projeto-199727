\documentclass[aspectratio=169]{beamer}
\usepackage[T1]{fontenc}
\usepackage{multicol}
\usepackage{ragged2e}   %new code
\usepackage[utf8]{inputenc}
\usepackage[brazil]{varioref, babel}
\usepackage{xmpmulti}
\usepackage{epsfig}
\usepackage{caption}
\usepackage{subcaption}
\usepackage{siunitx}
\usepackage{mathtools}
\usepackage{amssymb}
\usepackage{amsmath}
\usepackage{booktabs}
\usepackage{pbox}
\usepackage{graphicx,url}
\usepackage{etoolbox}
\usepackage{ru,hyperref,url} % 
\graphicspath{{../figures/}}
\usepackage{indentfirst}

\usepackage{tikz}
\usetikzlibrary{shapes,arrows,fit, positioning, arrows.meta}
\usetikzlibrary{backgrounds}
\usepgflibrary{shapes.multipart}

\addtobeamertemplate{block begin}{}{\justifying}
\setbeamertemplate{section in toc}[sections numbered]
\setbeamersize{text margin left = 1em}
\setbeamertemplate{caption}[numbered]
\setbeamertemplate{bibliography item}[text]

\newcommand{\doingles}[1]{\footnote{Do inglês: \emph{#1}.}}


\title{Computação Gráfica utilizando Redes Neurais Adversárias Generativas}

\subtitle{Computer Graphics with Generative Adversarial Neural Networks}

\author[João Pedro O. Pagnan]{ \footnotesize
  Aluno: João Pedro de Oliveira Pagnan\\
  Professora: Profa. Dra. Paula Dornhofer Paro Costa  [FEEC/UNICAMP]\\\medskip
  }

\institute[]{Universidade Estadual de Campinas - Faculdade de Engenharia Elétrica e Computação\\
  EA979 - Introdução à Computação Gráfica e ao Processamento de Imagem \\
  }

\date{\scriptsize \today}

\renewcommand{\indent}{\hspace*{2em}}

\begin{document}

\setbeamertemplate{headline}{}
\setbeamertemplate{footline}{}

\begin{frame}
  \titlepage
\end{frame}

\section{Introdução}
\begin{frame}
	\frametitle{Introdução}
	\justifying
	
	\vspace{7ex}	
	
	\indent{Com o crescimento do poderio computacional nos últimos anos, em especial, das placas de vídeo, o uso e a popularidade de modelos com redes neurais artificiais em tarefas de aprendizado de máquina, como predição, classificação, regressão etc., aumentou de maneira considerável.}
	
	\indent{Uma arquitetura de rede neural que ficou bastante conhecida foi a rede neural adversária generativa. Apresentada originalmente em 2014 por Ian Goodfellow e os outros membros do seu grupo de pesquisa \cite{goodfellow2014generative}, este modelo é capaz de, após um treinamento em um conjunto de dados, gerar novas amostras com os mesmos parâmetros das imagens que foram utilizados para treinar a rede.}
	
	\indent{Este tipo de rede neural é bastante utilizado, por exemplo, em tarefas de computação gráfica, sendo possível gerar fotos de rostos humanos, de animais, de artes, dentre vários outros exemplos.}	
	
	\vfill	
	
\end{frame}

\section{Objetivo do projeto}
\begin{frame}
    \frametitle{Objetivo do projeto}
    \justifying 
    
    \indent{Este projeto tem como objetivo implementar um modelo de rede neural adversário generativo de forma a estudar esta maneira de se realizar computação gráfica.}
    
    \indent{O modelo a ser implementado é a StyleGAN \cite{karras2019style}, apresentada em 2018 pelo NVlabs.}
    
    \indent{Esta GAN tornou-se famosa por causa do site }

\end{frame}

\section{O que são redes neurais?}
\begin{frame}
	\frametitle{O que são redes neurais?}
	\justifying
\end{frame}

\appendix

\begin{frame}{Referências}
    \tiny
    \bibliography{bib}
    \bibliographystyle{ieeetr}
\end{frame}

\begin{frame}[plain,c]
    \begin{center}
    \Huge Muito Obrigado!
    \end{center}
\end{frame}


\end{document}
